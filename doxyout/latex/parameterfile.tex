\-The parameterfile for \-G\-A\-D\-G\-E\-T-\/2 is a simple text file, consisting of pairs of tags and values. \-For each parameter, a separate line needs to be specified, first listing the name (tag) of the parameter, and then the assigned value, separated by whitespace. \-It is allowed to add further text behind the assigned parameter value. \-The order of the parameters is arbitrary, but each one needs to occur exactly one time, otherwise an error message will be produced. \-Empty lines, or lines beginning with a \%-\/sign, are ignored and treated as comments.


\begin{DoxyItemize}
\item {\bfseries \-Init\-Cond\-File} \par
 \-The filename of the initial conditions file. \-If a restart from a snapshot with the \char`\"{}2\char`\"{} option is desired, one needs to specify the snapshot file here.
\end{DoxyItemize}


\begin{DoxyItemize}
\item {\bfseries \-Output\-Dir} \par
 \-Pathname of the output directory of the code.
\end{DoxyItemize}


\begin{DoxyItemize}
\item {\bfseries \-Energy\-File} \par
 \-Filename of the log-\/file that contain the energy statistics.
\end{DoxyItemize}


\begin{DoxyItemize}
\item {\bfseries \-Info\-File} \par
 \-Log-\/file that contains a list of the timesteps taken.
\end{DoxyItemize}


\begin{DoxyItemize}
\item {\bfseries \-Timings\-File} \par
 \-Log-\/file with performance metrics of the gravitational tree computation.
\end{DoxyItemize}


\begin{DoxyItemize}
\item {\bfseries \-Cpu\-File} \par
 \-Log-\/file with \-C\-P\-U time consumption in various parts of the code.
\end{DoxyItemize}


\begin{DoxyItemize}
\item {\bfseries \-Restart\-File} \par
 \-Basename of restart-\/files produced by the code.
\end{DoxyItemize}


\begin{DoxyItemize}
\item {\bfseries \-Snapshot\-File\-Base} \par
 \-Basename of snapshot files produced by the code.
\end{DoxyItemize}


\begin{DoxyItemize}
\item {\bfseries \-Output\-List\-Filename} \par
 \-File with a list of the desired output times.
\end{DoxyItemize}


\begin{DoxyItemize}
\item {\bfseries \-Time\-Limit\-C\-P\-U} \par
 \-C\-P\-U-\/time limit for the present submission of the code. \-If 85 percent of this time have been reached at the end of a timestep, the code terminates itself and produces restart files.
\end{DoxyItemize}


\begin{DoxyItemize}
\item {\bfseries \-Resubmit\-On} \par
 \-If set to \char`\"{}1\char`\"{}, the code will try to resubmit itself to the queuing system when an interruption of the run due to the \-C\-P\-U-\/time limit occurs. \-The resubmission itself is done by executing the program/script given with {\itshape \-Resubmit\-Command\/}.
\end{DoxyItemize}


\begin{DoxyItemize}
\item {\bfseries \-Resubmit\-Command} \par
 \-The name of a script file or program that is executed for automatic resubmission of the job to the queuing system. \-Note that the file given here needs to be executable.
\end{DoxyItemize}


\begin{DoxyItemize}
\item {\bfseries \-I\-C\-Format} \par
 \-The file format of the initial conditions. \-Currently, three different formats are supported, selected by one of the choices \char`\"{}1\char`\"{}, \char`\"{}2\char`\"{}, or \char`\"{}3\char`\"{}. \-Format \char`\"{}1\char`\"{} is the traditional fortran-\/style unformatted format familiar from \-G\-A\-D\-G\-E\-T-\/1. \-Format \char`\"{}2\char`\"{} is a variant of this format, where each block of data is preceeded by a 4-\/character block-\/identifier. \-Finally, format \char`\"{}3\char`\"{} selects the \-H\-D\-F-\/5 format.
\end{DoxyItemize}


\begin{DoxyItemize}
\item {\bfseries \-Snap\-Format} \par
 \-Similar as {\itshape \-I\-C\-Format\/}, this parameter selects the file-\/format of snapshot dumps produced by the code.
\end{DoxyItemize}


\begin{DoxyItemize}
\item {\bfseries \-Comoving\-Integration\-On} \par
 \-If set to \char`\"{}1\char`\"{}, the code assumes that a cosmological integration in comoving coordinates is carried out, otherwise ordinary \-Newtonian dynamics is assumed.
\end{DoxyItemize}


\begin{DoxyItemize}
\item {\bfseries \-Type\-Of\-Timestep\-Criterion} \par
 \-This parameter can in principle be used to select different kinds of timestep criteria for gravitational dynamics. \-However, \-G\-A\-D\-G\-E\-T-\/2 presently only supports the standard criterion \char`\"{}0\char`\"{}.
\end{DoxyItemize}


\begin{DoxyItemize}
\item {\bfseries \-Output\-List\-On} \par
 \-If set to \char`\"{}1\char`\"{}, the code tries to read a list of desired output times from the file given in {\itshape \-Output\-List\-Filename\/}. \-Otherwise, output times are generated equally spaced from the values assigned for {\itshape \-Time\-Of\-First\-Snapshot\/} and {\itshape \-Time\-Bet\-Snapshot\/}.
\end{DoxyItemize}


\begin{DoxyItemize}
\item {\bfseries \-Periodic\-Boundaries\-On} \par
 \-If set to \char`\"{}1\char`\"{}, periodic boundary conditions are assumed, with a cubical box-\/size of side-\/length {\itshape \-Box\-Size\/}. \-Particle coordinates are expected to be in the range \mbox{[}0,{\itshape \-Box\-Size\/}\mbox{[}.
\end{DoxyItemize}


\begin{DoxyItemize}
\item {\bfseries \-Time\-Begin} \par
 \-This sets the starting time of a simulation when the code is started from initial conditions. \-For cosmological integrations, the value specified here is taken as the initial scale factor.
\end{DoxyItemize}


\begin{DoxyItemize}
\item {\bfseries \-Time\-Max} \par
 \-This sets the final time for the simulation. \-The code normally tries to run until this time is reached. \-For cosmological integrations, the value given here is the final scale factor.
\end{DoxyItemize}


\begin{DoxyItemize}
\item {\bfseries \-Omega0} \par
 \-Gives the total matter density (in units of the critical density) at z=0 for cosmological simulations.
\end{DoxyItemize}


\begin{DoxyItemize}
\item {\bfseries \-Omega\-Lambda} \par
 \-Gives the vacuum energy density at z=0 for cosmological simulations.
\end{DoxyItemize}


\begin{DoxyItemize}
\item {\bfseries \-Omega\-Baryon} \par
 \-Gives the baryon density at z=0 for cosmological simulations.
\end{DoxyItemize}


\begin{DoxyItemize}
\item {\bfseries \-Hubble\-Param} \par
 \-This gives the \-Hubble constant at z=0 in units of 100 km/sec/\-Mpc. \-Note that this parameter has been basically absorbed into the definition of the internal code units, such that for gravitational dynamics and adiabatic gas dynamics the actual value assigned for {\itshape \-Hubble\-Param\/} is not used by the code.
\end{DoxyItemize}


\begin{DoxyItemize}
\item {\bfseries \-Box\-Size} \par
 \-The boxsize for simulations with periodic boundary conditions.
\end{DoxyItemize}


\begin{DoxyItemize}
\item {\bfseries \-Time\-Of\-First\-Snapshot} \par
 \-The time of the first desired snapshot file in case a file with output times is not specified. \-For cosmological simulations, the value given here is the scale factor of the first desired output.
\end{DoxyItemize}


\begin{DoxyItemize}
\item {\bfseries \-Time\-Bet\-Snapshot} \par
 \-The time interval between two subsequent snapshot files in case a file with output times is not specified. \-For cosmological simulations, this is a multiplicative factor applied to the time of the last snapshot, such that the snapshots will have a constant logarithmic spacing in the scale factor. \-Otherwise, the parameter is an additive constant that gives the linear spacing between snapshot times.
\end{DoxyItemize}


\begin{DoxyItemize}
\item {\bfseries \-Cpu\-Time\-Bet\-Restart\-File} \par
 \-The value specfied here gives the time in seconds the code will run before it writes regularly produced restart files. \-This can be useful to protect against unexpected interruptions (for example due to a hardware problem) of a simulation, particularly if it is run for a long time. \-It is then possible to resume a simulation from the last restart file, reducing the potential loss to the elapsed \-C\-P\-U-\/time since this was produced.
\end{DoxyItemize}


\begin{DoxyItemize}
\item {\bfseries \-Time\-Bet\-Statistics} \par
 \-The code can be asked to measure the total kinetic, thermal, and potential energy in regular intervals, and to write the results to the file given in {\itshape \-Energy\-File\/}. \-The time interval between two such measurements is given by the parameter {\itshape \-Time\-Bet\-Statistics\/}, in an analogous way as with {\itshape \-Time\-Bet\-Snapshot\/}. \-Note that the compile time option {\itshape \-C\-O\-M\-P\-U\-T\-E\-\_\-\-P\-O\-T\-E\-N\-T\-I\-A\-L\-\_\-\-E\-N\-E\-R\-G\-Y\/} needs to be activated to obtain a measurement of the gravitational potential energy.
\end{DoxyItemize}


\begin{DoxyItemize}
\item {\bfseries \-Num\-Files\-Per\-Snapshot} \par
 \-The number of separate files requested for each snapshot dump. \-Each file of the snapshot will hold the data of one or several processors, up to all of them. {\itshape \-Num\-Files\-Per\-Snapshot\/} must hence lie between 1 and the number of processors used. \-Distributing a snapshot onto several files can be done in parallel and may lead to much better \-I/\-O performance, depending on the hardware configuration. \-It can also help to avoid problems due to big files ($>$2\-G\-B) for large simulations. \-Note that initial conditions may also be distributed into several files, the number of which is automatically recognised by the code and does not have to be equal to {\itshape \-Num\-Files\-Per\-Snapshot\/} (it may also be larger than the number of processors).
\end{DoxyItemize}


\begin{DoxyItemize}
\item {\bfseries \-Num\-Files\-Written\-In\-Parallel} \par
 \-The number of files the code may read or write simultaneously when writing or reading snapshot/restart files. \-The value of this parameter must be smaller or equal to the number of processors.
\end{DoxyItemize}


\begin{DoxyItemize}
\item {\bfseries \-Err\-Tol\-Int\-Accuracy} \par
 \-This dimensionless parameter controls the accuracy of the timestep criterion selected by {\itshape \-Type\-Of\-Timestep\-Criterion\/}.
\end{DoxyItemize}


\begin{DoxyItemize}
\item {\bfseries \-Courant\-Fac} \par
 \-This sets the value of the \-Courant parameter used in the determination of the hydrodynamical timestep of \-S\-P\-H particles.
\end{DoxyItemize}


\begin{DoxyItemize}
\item {\bfseries \-Max\-Size\-Timestep} \par
 \-This gives the maximum timestep a particle may take. \-This should be set to a sensible value in order to protect against too large timesteps for particles with very small acceleration. \-For cosmological simulations, the parameter given here is the maximum allowed step in the logarithm of the expansion factor. \-Note that the definition of \-Max\-Size\-Timestep has {\itshape changed\/} compared to \-Gadget-\/1.\-1 for cosmological simulations.
\end{DoxyItemize}


\begin{DoxyItemize}
\item {\bfseries \-Min\-Size\-Timestep} \par
 \-If a particle requests a timestep smaller than the value specified here, the code will normally terminate with a warning message. \-If compiled with the {\itshape \-N\-O\-S\-T\-O\-P\-\_\-\-W\-H\-E\-N\-\_\-\-B\-E\-L\-O\-W\-\_\-\-M\-I\-N\-T\-I\-M\-E\-S\-T\-E\-P\/} option, the code will instead force the timesteps to be at least as large as {\itshape \-Min\-Size\-Timestep\/}.
\end{DoxyItemize}


\begin{DoxyItemize}
\item {\bfseries \-Type\-Of\-Opening\-Criterion} \par
 \-This selects the type of cell-\/opening criterion used in the tree walks. \-A value of `0' results in standard \-Barnes \& \-Hut, while `1' selects the relative opening criterion of \-G\-A\-D\-G\-E\-T-\/2.
\end{DoxyItemize}


\begin{DoxyItemize}
\item {\bfseries \-Err\-Tol\-Theta} \par
 \-This gives the maximum opening angle if the \-B\-H criterion is used for the tree walk. \-If the relative opening criterion is used instead, a first force estimate is computed using the \-B\-H algorithm, which is then recomputed with the relative opening criterion.
\end{DoxyItemize}


\begin{DoxyItemize}
\item {\bfseries \-Err\-Tol\-Force\-Acc} \par
 \-The accuracy parameter for the relative opening criterion for the tree walk.
\end{DoxyItemize}


\begin{DoxyItemize}
\item {\bfseries \-Tree\-Domain\-Update\-Frequency} \par
 \-The domain decomposition and tree construction need not necessarily be done every single timestep. \-Instead, tree nodes can be dynamically updated, which is faster. \-However, the tree walk will become more expensive since the tree nodes have to \char`\"{}grow\char`\"{} to keep accomodating all particles they enclose. \-The parameter {\itshape \-Tree\-Domain\-Update\-Frequency\/} controls how often the domain decomposition is carried out and the tree is reconstructed from scratch. \-For example, a value of 0.\-1 means that the domain decomposition and the tree are reconstructed whenever there have been more than 0.\-1$\ast$\-N force computations since the last reconstruction, where \-N is the total particle number. \-A value of 0 will reconstruct the tree every timestep.
\end{DoxyItemize}


\begin{DoxyItemize}
\item {\bfseries \-Max\-R\-M\-S\-Displacement\-Fac} \par
 \-This parameter is an additional timestep criterion for the long-\/range integration in case the \-Tree\-P\-M algorithm is used. \-It limits the long-\/range timestep such that the rms-\/displacement of particles per step is at most {\itshape \-Max\-R\-M\-S\-Displacement\-Fac\/} times the mean particle separation, or the mesh-\/scale, whichever is smaller.
\end{DoxyItemize}


\begin{DoxyItemize}
\item {\bfseries \-Des\-Num\-Ngb} \par
 \-This sets the desired number of \-S\-P\-H smoothing neighbours.
\end{DoxyItemize}


\begin{DoxyItemize}
\item {\bfseries \-Max\-Num\-Ngb\-Deviation} \par
 \-This sets the allowed variation of the number of neighbours around the target value {\itshape \-Des\-Num\-Ngb\/}.
\end{DoxyItemize}


\begin{DoxyItemize}
\item {\bfseries \-Art\-Bulk\-Visc\-Const} \par
 \-This sets the value of the artificial viscosity parameter used by \-G\-A\-D\-G\-E\-T-\/2.
\end{DoxyItemize}


\begin{DoxyItemize}
\item {\bfseries \-Init\-Gas\-Temp} \par
 \-This sets the initial gas temperature (assuming either a mean molecular weight corresponding to full ionization or full neutrality, depending on whether the temperature is above or below 10$^\wedge$4 \-K) in \-Kelvin when initial conditions are read. \-However, the gas temperature is only set to a certain temperature if {\itshape \-Init\-Gas\-Temp\/}$>$0, and if the temperature of the gas particles in the initial conditions file is zero, otherwise the initial gas temperature is left at the value stored in the \-I\-C file.
\end{DoxyItemize}


\begin{DoxyItemize}
\item {\bfseries \-Min\-Gas\-Temp} \par
 \-A minimum temperature floor imposed by the code. \-This may be set to zero.
\end{DoxyItemize}


\begin{DoxyItemize}
\item {\bfseries \-Part\-Alloc\-Factor} \par
 \-Each processor allocates space for {\itshape \-Part\-Alloc\-Factor\/} times the average number of particles per processor. \-This number needs to be larger than 1 to allow the simulation to achieve a good work-\/load balancing, which requires to trade particle-\/load balance for work-\/load balance. \-It is good to make {\itshape \-Part\-Alloc\-Factor\/} quite a bit larger than 1, but values in excess of 3 will typically not improve performance any more. \-For a value that is too small, the code may not be able to succeed in the domain decomposition and terminate.
\end{DoxyItemize}


\begin{DoxyItemize}
\item {\bfseries \-Tree\-Alloc\-Factor} \par
 \-To construct the \-B\-H-\/tree for \-N particles, somewhat less than \-N internal tree-\/nodes are necessary for `normal' particle distributions. {\itshape \-Tree\-Alloc\-Factor\/} sets the number of internal tree-\/nodes allocated in units of the particle number. \-By experience, space for 0.\-65 \-N internal nodes is usually fully sufficient, so a value of 0.\-7 should put you on the safe side.
\end{DoxyItemize}


\begin{DoxyItemize}
\item {\bfseries \-Buffer\-Size} \par
 \-This specifies the size (in \-M\-Byte per processor) of a communication buffer used by the code.
\end{DoxyItemize}


\begin{DoxyItemize}
\item {\bfseries \-Unit\-Length\-\_\-in\-\_\-cm} \par
 \-This sets the internal length unit in cm/h, where \-H\-\_\-0 = 100 h km/sec/\-Mpc. \-For example, a choice of 3.\-085678e21 sets the length unit to 1.\-0 kpc/h.
\end{DoxyItemize}


\begin{DoxyItemize}
\item {\bfseries \-Unit\-Mass\-\_\-in\-\_\-g} \par
 \-This sets the internal mass unit in g/h, where \-H\-\_\-0 = 100 h km/sec/\-Mpc. \-For example, a choice of 1.\-989e43 sets the mass unit to 10$^\wedge$10 \-M\-\_\-sun/h.
\end{DoxyItemize}


\begin{DoxyItemize}
\item {\bfseries \-Unit\-Velocity\-\_\-in\-\_\-cm\-\_\-per\-\_\-s} \par
 \-This sets the internal velocity unit in cm/sec. \-For example, a choice of 1e5 sets the velocity unit to km/sec. \-Note that the specification of {\itshape \-Unit\-Length\-\_\-in\-\_\-cm\/}, {\itshape \-Unit\-Mass\-\_\-in\-\_\-g\/}, and {\itshape \-Unit\-Velocity\-\_\-in\-\_\-cm\-\_\-per\-\_\-s\/} also determines the internal unit of time.
\end{DoxyItemize}


\begin{DoxyItemize}
\item {\bfseries \-Gravity\-Constant\-Internal} \par
 \-The numerical value of the gravitational constant \-G in internal units depends on the system of units you choose. \-For example, for the choices above, \-G=43007.\-1 in internal units. \-For {\itshape \-Gravity\-Constant\-Internal\/}=0, the code calculates the value corresponding to the physical value of \-G automatically. \-However, you might want to set \-G yourself. \-For example, by specifying {\itshape \-Gravity\-Constant\-Internal\/}=1, {\itshape \-Unit\-Length\-\_\-in\-\_\-cm\/}=1, {\itshape \-Unit\-Mass\-\_\-in\-\_\-g\/}=1, and {\itshape \-Unit\-Velocity\-\_\-in\-\_\-cm\-\_\-per\-\_\-s\/}=1, one obtains a `natural' system of units. \-Note that the code will nevertheless try to use the `correct' value of the \-Hubble constant in this case, so you should not set {\itshape \-Gravity\-Constant\-Internal\/} in cosmological integrations.
\end{DoxyItemize}


\begin{DoxyItemize}
\item {\bfseries \-Min\-Gas\-Hsml\-Fractional} \par
 \-This parameter sets the minimum allowed \-S\-P\-H smoothing length in units of the gravitational softening length of the gas particles. \-The smoothing length will be prevented from falling below this value. \-When this bound is actually reached, the number of smoothing neighbors will instead be increased above {\itshape \-Des\-Num\-Ngb\/}.
\end{DoxyItemize}


\begin{DoxyItemize}
\item {\bfseries \-Softening\-Gas} \par
 \-The \-Plummer equivalent gravitational softening length for particle type 0, which are the gas particles. \-For cosmological simulations in comoving coordinates, this is interpreted as a comoving softening length.
\end{DoxyItemize}


\begin{DoxyItemize}
\item {\bfseries \-Softening\-Halo} \par
 \-The \-Plummer equivalent gravitational softening length for particle type 1.
\end{DoxyItemize}


\begin{DoxyItemize}
\item {\bfseries \-Softening\-Disk} \par
 \-The \-Plummer equivalent gravitational softening length for particle type 2.
\end{DoxyItemize}


\begin{DoxyItemize}
\item {\bfseries \-Softening\-Bulge} \par
 \-The \-Plummer equivalent gravitational softening length for particle type 3.
\end{DoxyItemize}


\begin{DoxyItemize}
\item {\bfseries \-Softening\-Stars} \par
 \-The \-Plummer equivalent gravitational softening length for particle type 4.
\end{DoxyItemize}


\begin{DoxyItemize}
\item {\bfseries \-Softening\-Bndry} \par
 \-The \-Plummer equivalent gravitational softening length for particle type 5.
\end{DoxyItemize}


\begin{DoxyItemize}
\item {\bfseries \-Softening\-Gas\-Max\-Phys} \par
 \-When comoving integration is used, this parameter gives the maximum physical gravitational softening length for particle type 0. \-Depening on the relative settings of {\itshape \-Softening\-Gas\/} and {\itshape \-Softening\-Gas\-Max\-Phys\/}, the code will hence switch from a softening constant in comoving units to one constant in physical units.
\end{DoxyItemize}


\begin{DoxyItemize}
\item {\bfseries \-Softening\-Halo\-Max\-Phys} \par
 \-When comoving integration is used, this parameter gives the maximum physical gravitational softening length for particle type 1.
\end{DoxyItemize}


\begin{DoxyItemize}
\item {\bfseries \-Softening\-Disk\-Max\-Phys} \par
 \-When comoving integration is used, this parameter gives the maximum physical gravitational softening length for particle type 2.
\end{DoxyItemize}


\begin{DoxyItemize}
\item {\bfseries \-Softening\-Bulge\-Max\-Phys} \par
 \-When comoving integration is used, this parameter gives the maximum physical gravitational softening length for particle type 3.
\end{DoxyItemize}


\begin{DoxyItemize}
\item {\bfseries \-Softening\-Stars\-Max\-Phys} \par
 \-When comoving integration is used, this parameter gives the maximum physical gravitational softening length for particle type 4.
\end{DoxyItemize}


\begin{DoxyItemize}
\item {\bfseries \-Softening\-Bndry\-Max\-Phys} \par
 \-When comoving integration is used, this parameter gives the maximum physical gravitational softening length for particle type 5. 
\end{DoxyItemize}